% Hopf Algebras and Combinatorics

\documentclass[svgnames]{article}
\usepackage[paperwidth=6in, paperheight=8in, top = 20mm, bottom = 18mm, left=10mm, right = 10mm]{geometry}

\usepackage{natbib}

\usepackage{amsmath, amsfonts, amssymb, amsthm}
\usepackage[all, cmtip, 2cell]{xy}
\setcounter{tocdepth}{3}
\usepackage{graphicx}
\usepackage{physics}
\usepackage{tikz}
\usepackage{mathtools}
\usepackage{xspace}
\usepackage{enumitem}

\usepackage{tocloft}
\renewcommand{\cftdot}{}

\usepackage{hyperref}
\hypersetup{colorlinks, linkcolor = [RGB]{66, 128, 128}, urlcolor = red, linktocpage = true}


\usepackage{newpxmath}
\usepackage{charter}
\usepackage[T1]{fontenc}


\newtheorem{Theorem}{Theorem}
\newtheorem{Lemma}{Lemma}

\theoremstyle{definition}
\newtheorem{Definition}{Definition}
\newtheorem*{Definition*}{Definition}
\newtheorem{Example}{Example}
\newtheorem*{Example*}{Example}
\newtheorem{Exercise}{Exercise}

\theoremstyle{remark}
\newtheorem*{Remark*}{Remark}
\newtheorem*{Solution*}{Solution}
\newtheorem*{Note*}{Note}


\newtheoremstyle{underline}% name
{}        % Space above, empty = `usual value'
{}              % Space below
{}              % Body font
{}    % Indent amount (empty = no indent, \parindent = para indent)
{}              % Thm head font
{:}             % Punctuation after thm head
{1.5mm}         % Space after thm head: \newline = linebreak
{\underline{\thmname{#1}\thmnumber{ #2}\thmnote{(#3)}}}  % Thm head spec

\theoremstyle{underline}
\newtheorem*{Multiplication*}{Multiplication}

\theoremstyle{underline}
\newtheorem*{Comultiplication*}{Comultiplication}

\DeclareMathOperator{\Mat}{Mat}
\DeclareMathOperator{\Hilb}{Hilb}
\newcommand{\Mod}[1]{\ (\mathrm{mod}\ #1)}

\let\Im\relax
\DeclareMathOperator{\Im}{Im}
\newcommand{\id}{\mathrm{id}}
\newcommand{\Hom}{\mathrm{Hom}}
\newcommand{\Iso}{\mathrm{Iso}}
\renewcommand{\th}{\textsuperscript{th}\xspace}

\newlist{subquests}{enumerate}{2}
\setlist[subquests, 1]{label = (\alph*)}
\setlist[subquests, 2]{label = \roman*.}

\setlength{\parindent}{0pt}


\begin{document}
	
	\title{\textbf{Heaps and Applications}}

	\author{\small Arjit Seth \\ \small Manipal Institute of Technology, Manipal University}
	\date{}
	\maketitle
	
	\renewcommand{\abstractname}{Context}
	\begin{abstract}
	\noindent These are notes created based on a talk presented by Dr. K. N. Raghavan from The Institute of Mathematical Sciences, Chennai at Computer Applications based on Modern Algebra Conference at Manipal Institute of Technology, Manipal University. The talk covers an introduction to the chromatic polynomial and its relation to acyclic orientations, Stanley's theorem and an informal introduction to heaps.  
	\end{abstract}
	
	\begingroup
	\let\clearpage\relax
	\tableofcontents
	\endgroup

	\section{Chromatic Polynomial}
	George David Birkhoff attacked the four color problem by constructing a polynomial which characterises the number of vertex colourings, called the \emph{chromatic polynomial}. The motivation is provided by example:
	\begin{Example*}
		Let $G$ be the following graph:
		\begin{gather*}
			\tikz[baseline=(current bounding box.center),scale=1,auto=left,every node/.style={circle,draw=black,thin,inner sep=0pt,minimum size=2mm}]
				{
					\node (n4) at (5,1)	{};
					\node (n5) at (5,0)	{};
					\node (n6) at (6,1)	{};
					\node (n7) at (6,0)	{};
					\foreach \from/\to in {n4/n5, n4/n6, n4/n7, n5/n7, n6/n7}
					\draw (\from) -- (\to);
			}
		\end{gather*}
		The minimum number of colours required to colour this graph is 3 because of the adjacency of two diagonal vertices. There are 6 ways of colouring this graph with 3 colours:
		\begin{gather*}
			\tikz[baseline=(current bounding box.center),scale=1,auto=left,every node/.style={circle,inner sep=0pt,minimum size=2mm}]
				{
					\node[fill=red] (n4) at (5,1)	{};
					\node[fill=blue] (n5) at (5,0)	{};
					\node[fill=blue] (n6) at (6,1)	{};
					\node[fill=green] (n7) at (6,0)	{};
					\foreach \from/\to in {n4/n5, n4/n6, n4/n7, n5/n7, n6/n7}
					\draw (\from) -- (\to);
			} \hspace{12pt} 
			\tikz[baseline=(current bounding box.center),scale=1,auto=left,every node/.style={circle,draw=black,thin,inner sep=0pt,minimum size=2mm}]
				{
					\node (n4) at (5,1)	{};
					\node (n5) at (5,0)	{};
					\node (n6) at (6,1)	{};
					\node (n7) at (6,0)	{};
					\foreach \from/\to in {n4/n5, n4/n6, n4/n7, n5/n7, n6/n7}
					\draw (\from) -- (\to);
			} \hspace{12pt}
			\tikz[baseline=(current bounding box.center),scale=1,auto=left,every node/.style={circle,draw=black,thin,inner sep=0pt,minimum size=2mm}]
				{
					\node (n4) at (5,1)	{};
					\node (n5) at (5,0)	{};
					\node (n6) at (6,1)	{};
					\node (n7) at (6,0)	{};
					\foreach \from/\to in {n4/n5, n4/n6, n4/n7, n5/n7, n6/n7}
					\draw (\from) -- (\to);
			} \hspace{12pt}
			\tikz[baseline=(current bounding box.center),scale=1,auto=left,every node/.style={circle,draw=black,thin,inner sep=0pt,minimum size=2mm}]
				{
					\node (n4) at (5,1)	{};
					\node (n5) at (5,0)	{};
					\node (n6) at (6,1)	{};
					\node (n7) at (6,0)	{};
					\foreach \from/\to in {n4/n5, n4/n6, n4/n7, n5/n7, n6/n7}
					\draw (\from) -- (\to);
			} \hspace{12pt} 
			\tikz[baseline=(current bounding box.center),scale=1,auto=left,every node/.style={circle,draw=black,thin,inner sep=0pt,minimum size=2mm}]
				{
					\node (n4) at (5,1)	{};
					\node (n5) at (5,0)	{};
					\node (n6) at (6,1)	{};
					\node (n7) at (6,0)	{};
					\foreach \from/\to in {n4/n5, n4/n6, n4/n7, n5/n7, n6/n7}
					\draw (\from) -- (\to);
			} \hspace{12pt}
			\tikz[baseline=(current bounding box.center),scale=1,auto=left,every node/.style={circle,draw=black,thin,inner sep=0pt,minimum size=2mm}]
				{
					\node (n4) at (5,1)	{};
					\node (n5) at (5,0)	{};
					\node (n6) at (6,1)	{};
					\node (n7) at (6,0)	{};
					\foreach \from/\to in {n4/n5, n4/n6, n4/n7, n5/n7, n6/n7}
					\draw (\from) -- (\to);
			}
		\end{gather*}
	\end{Example*}
	\begin{Definition}[Chromatic Polynomial]
		The \emph{chromatic polynomial} $\gamma_G(\lambda)$ of a graph $G$ counts the number of its proper vertex colourings.
	\end{Definition}


\end{document}