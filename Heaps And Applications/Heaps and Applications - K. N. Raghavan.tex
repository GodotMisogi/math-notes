% Hopf Algebras and Combinatorics

\documentclass[svgnames]{article}
\usepackage[paperwidth=6in, paperheight=8in, top = 20mm, bottom = 18mm, left=10mm, right = 10mm]{geometry}

\usepackage{natbib}

\usepackage{amsmath, amsfonts, amssymb, amsthm}
\usepackage[all, cmtip, 2cell]{xy}
\setcounter{tocdepth}{3}
\usepackage{graphicx}
\usepackage{physics}
\usepackage{tikz}
\usepackage{mathtools}
\usepackage{xspace}
\usepackage{enumitem}
\usepackage{url}
\usepackage{hyperref}

\usepackage{tocloft}
\renewcommand{\cftdot}{}

\usepackage{hyperref}
\hypersetup{colorlinks, linkcolor = [RGB]{66, 128, 128}, urlcolor = darkgray, linktocpage = true}


\usepackage{newpxmath}
\usepackage{charter}
\usepackage[T1]{fontenc}


\newtheorem{Theorem}{Theorem}
\newtheorem{Lemma}{Lemma}

\theoremstyle{definition}
\newtheorem{Definition}{Definition}
\newtheorem*{Definition*}{Definition}
\newtheorem{Example}{Example}
\newtheorem*{Example*}{Example}
\newtheorem{Exercise}{Exercise}

\theoremstyle{remark}
\newtheorem*{Remark*}{Remark}
\newtheorem*{Solution*}{Solution}
\newtheorem*{Note*}{Note}


\newtheoremstyle{underline}% name
{}        % Space above, empty = `usual value'
{}              % Space below
{}              % Body font
{}    % Indent amount (empty = no indent, \parindent = para indent)
{}              % Thm head font
{:}             % Punctuation after thm head
{1.5mm}         % Space after thm head: \newline = linebreak
{\underline{\thmname{#1}\thmnumber{ #2}\thmnote{(#3)}}}  % Thm head spec

\theoremstyle{underline}
\newtheorem*{Multiplication*}{Multiplication}

\theoremstyle{underline}
\newtheorem*{Comultiplication*}{Comultiplication}

\DeclareMathOperator{\Mat}{Mat}
\DeclareMathOperator{\Hilb}{Hilb}
\newcommand{\Mod}[1]{\ (\mathrm{mod}\ #1)}

\let\Im\relax
\DeclareMathOperator{\Im}{Im}
\newcommand{\id}{\mathrm{id}}
\newcommand{\Hom}{\mathrm{Hom}}
\newcommand{\Iso}{\mathrm{Iso}}
\renewcommand{\th}{\textsuperscript{th}\xspace}

\newlist{subquests}{enumerate}{2}
\setlist[subquests, 1]{label = (\alph*)}
\setlist[subquests, 2]{label = \roman*.}

\setlength{\parindent}{0pt}


\begin{document}
	
	\title{\textbf{Heaps and Applications}}

	\author{\small Arjit Seth \\ \small Manipal Institute of Technology, Manipal University}
	\date{}
	\maketitle
	
	\renewcommand{\abstractname}{Context}
	\begin{abstract}
	\noindent These are notes created based on a talk presented by Dr. K. N. Raghavan from The Institute of Mathematical Sciences (IMSc), Chennai at the \emph{Computer Applications based on Modern Algebra} conference at Manipal Institute of Technology. The talk was mostly for publicity of Xavier Viennot's \href{http://www.xavierviennot.org/coursIMSc2017/contents.html}{lectures} at IMSc, uploaded on the \href{https://www.youtube.com/watch?v=vRTdLDnpIt0&list=PLhkiT_RYTEU2HhH2ljHjA-M6v2FMcLytO}{\texttt{matscience}} YouTube channel. The talk covers an introduction to the chromatic polynomial and its relation to acyclic orientations, Stanley's theorem and an informal introduction to heaps.  
	\end{abstract}
	
	\begingroup
	\let\clearpage\relax
	\tableofcontents
	\endgroup

	\section{Chromatic Polynomial}
	George David Birkhoff attacked the four color problem by constructing a polynomial which characterises the number of vertex colourings for a variable number of colours, called the \emph{chromatic polynomial}. The motivation is provided by example:
	\begin{Example*}
		Let $G$ be the following graph:
		\begin{gather*}
			\tikz[baseline=(current bounding box.center),scale=1,auto=left,every node/.style={circle,draw=black,thick,inner sep=0pt,minimum size=2mm}]
			{
				\node (n1) at (0,1)	{};
				\node (n2) at (0,0)	{};
				\node (n3) at (1,1)	{};
				\node (n4) at (1,0)	{};
				\foreach \from/\to in {n1/n2, n1/n3, n1/n4, n2/n4, n3/n4}
				\draw[thick] (\from) -- (\to);
			}
		\end{gather*}
		The minimum number of colours required to colour this graph is 3 because of the two diagonally adjacent vertices. There are 6 ways of colouring this graph with 3 colours:
		\begin{gather*}
			\tikz[baseline=(current bounding box.center),scale=1,auto=left,every node/.style={circle,inner sep=0pt,minimum size=2mm}]
			{
				\node[fill=red!80] (n1) at (0,1)	{};
				\node[fill=blue!80] (n2) at (0,0)	{};
				\node[fill=blue!80] (n3) at (1,1)	{};
				\node[fill=green!80] (n4) at (1,0)	{};
				\foreach \from/\to in {n1/n2, n1/n3, n1/n4, n2/n4, n3/n4}
				\draw[thick] (\from) -- (\to);
			} \hspace{12pt} 
			\tikz[baseline=(current bounding box.center),scale=1,auto=left,every node/.style={circle,inner sep=0pt,minimum size=2mm}]
			{
				\node[fill=blue!80] (n1) at (0,1)	{};
				\node[fill=green!80] (n2) at (0,0)	{};
				\node[fill=green!80] (n3) at (1,1)	{};
				\node[fill=red!80] (n4) at (1,0)	{};
				\foreach \from/\to in {n1/n2, n1/n3, n1/n4, n2/n4, n3/n4}
				\draw[thick] (\from) -- (\to);
			} \hspace{12pt}
			\tikz[baseline=(current bounding box.center),scale=1,auto=left,every node/.style={circle,inner sep=0pt,minimum size=2mm}]
			{
				\node[fill=green!80] (n1) at (0,1)	{};
				\node[fill=blue!80] (n2) at (0,0)	{};
				\node[fill=blue!80] (n3) at (1,1)	{};
				\node[fill=red!80] (n4) at (1,0)	{};
				\foreach \from/\to in {n1/n2, n1/n3, n1/n4, n2/n4, n3/n4}
				\draw[thick] (\from) -- (\to);
			} \hspace{12pt}
			\tikz[baseline=(current bounding box.center),scale=1,auto=left,every node/.style={circle,inner sep=0pt,minimum size=2mm}]
			{
				\node[fill=red!80] (n1) at (0,1)	{};
				\node[fill=green!80] (n2) at (0,0)	{};
				\node[fill=green!80] (n3) at (1,1)	{};
				\node[fill=blue!80] (n4) at (1,0)	{};
				\foreach \from/\to in {n1/n2, n1/n3, n1/n4, n2/n4, n3/n4}
				\draw[thick] (\from) -- (\to);
			} \hspace{12pt} 
			\tikz[baseline=(current bounding box.center),scale=1,auto=left,every node/.style={circle,inner sep=0pt,minimum size=2mm}]
			{
				\node[fill=blue!80] (n1) at (0,1)	{};
				\node[fill=red!80] (n2) at (0,0)	{};
				\node[fill=red!80] (n3) at (1,1)	{};
				\node[fill=green!80] (n4) at (1,0)	{};
				\foreach \from/\to in {n1/n2, n1/n3, n1/n4, n2/n4, n3/n4}
				\draw[thick] (\from) -- (\to);
			} \hspace{12pt}
			\tikz[baseline=(current bounding box.center),scale=1,auto=left,every node/.style={circle,inner sep=0pt,minimum size=2mm}]
			{
				\node[fill=green!80] (n1) at (0,1)	{};
				\node[fill=red!80] (n2) at (0,0)	{};
				\node[fill=red!80] (n3) at (1,1)	{};
				\node[fill=blue!80] (n4) at (1,0)	{};
				\foreach \from/\to in {n1/n2, n1/n3, n1/n4, n2/n4, n3/n4}
				\draw[thick] (\from) -- (\to);
			}
		\end{gather*}
		An exact colouring requires use of all $\lambda$ colours in the graph, called a $\lambda$-colouring. There are obviously 4! exact 4-colourings, since each vertex is coloured differently. To obtain a general formula for $\lambda$-colourings, consider the following argument: There are $\lambda$ colours one can use for the top left vertex and $\lambda - 1$ colours for the bottom right vertex; these conditions immediately determine that the other two vertices must be of the same colour with $\lambda - 2$ colours for them. Therefore, the general number of $\lambda$-colourings for this graph is given by the polynomial:
		\begin{gather*}
			\lambda\pqty{\lambda -1}\pqty{\lambda-2}^2
		\end{gather*}
		This is the chromatic polynomial of the graph, which will be defined shortly. Using this formula, the total number of 4-colourings (including ways of colouring the graph using 3 colours) is 48.
	\end{Example*}
	\begin{Definition}[Chromatic Polynomial]
		The \emph{chromatic polynomial} $\gamma_G(\lambda)$ of a graph $G$ counts the number of its proper vertex colourings with $\lambda$ colours. Its general formula is given by:
		\begin{gather*}
			\gamma_G\pqty{\lambda} = \sum_{k \geq 1} \gamma'_{G}\pqty{k}\binom{\lambda}{k}
		\end{gather*}
		where $\gamma'_{G}\pqty{k}$ denotes the number of exact k-colourings of $G$.
	\end{Definition}
	\begin{Remark*}
		Note that the number of vertices $n$ of the graph need not be mentioned since $\gamma'_{G}\pqty{k} = 0$ if $k > n$, obviously.
	\end{Remark*}
	As is customary in any investigation, inserting `forbidden' values into the polynomial, such as negative integers, is good experimentation. Let's try this for the previous graph:
	\begin{gather*}
		\gamma_G\pqty{-1} = 6 \times \binom{-1}{3} + 24 \times \binom{-1}{4} = 18 
	\end{gather*}
	Surprisingly enough, this the number of different acyclic orientations of the graph!

	\section{Acyclic Orientations}

	\begin{Definition}[Acyclic Orientation]
		An orientation of each edge of a graph $G$ such that no cycle in the graph is a cycle in the resulting directed graph.
	\end{Definition}
	\begin{Example*}
		Three different acyclic orientations of the previous graph are:
		\begin{gather*}
			\tikz[baseline=(current bounding box.center),scale=1,auto=left,every node/.style={circle,draw=black,thick,inner sep=0pt,minimum size=2mm}]
			{
				\node (n1) at (0,1)	{};
				\node (n2) at (0,0)	{};
				\node (n3) at (1,1)	{};
				\node (n4) at (1,0)	{};
				\foreach \from/\to in {n1/n2, n3/n1, n4/n1, n4/n2, n4/n3}
				\draw[->,thick] (\from) -- (\to);
			} \hspace{20pt}
			\tikz[baseline=(current bounding box.center),scale=1,auto=left,every node/.style={circle,draw=black,thick,inner sep=0pt,minimum size=2mm}]
			{
				\node (n1) at (0,1)	{};
				\node (n2) at (0,0)	{};
				\node (n3) at (1,1)	{};
				\node (n4) at (1,0)	{};
				\foreach \from/\to in {n2/n1, n3/n1, n4/n1, n4/n2, n4/n3}
				\draw[->,thick] (\from) -- (\to);
			} \hspace{20pt}
			\tikz[baseline=(current bounding box.center),scale=1,auto=left,every node/.style={circle,draw=black,thick,inner sep=0pt,minimum size=2mm}]
			{
				\node (n1) at (0,1)	{};
				\node (n2) at (0,0)	{};
				\node (n3) at (1,1)	{};
				\node (n4) at (1,0)	{};
				\foreach \from/\to in {n2/n1, n3/n1, n4/n1, n2/n4, n3/n4}
				\draw[->,thick] (\from) -- (\to);
			}
		\end{gather*}
	\end{Example*}
	\begin{Example*}
		Let $T$ be the following \emph{tree}:
		\begin{gather*}
			\tikz[baseline=(current bounding box.center),scale=1,auto=left,every node/.style={circle,draw=black,thick,inner sep=0pt,minimum size=2mm}]
			{
				\node (n1) at (0,0) {};
				\node (n2) at (1,0)	{};
				\node (n3) at (1,1)	{};
				\node (n4) at (2,0)	{};
				\node (n5) at (2,-1) {};
				\node (n6) at (3,0) {};
				\foreach \from/\to in {n1/n2, n2/n3, n2/n4, n4/n5, n4/n6}
				\draw[thick] (\from) -- (\to);
			} 	
		\end{gather*}
		$\lambda$ colours can be used to colour the first vertex, and $\lambda - 1$ colours can be used for the rest of them. Therefore, the chromatic polynomial of $T$ and the number of its acyclic orientations are:
		\begin{gather*}
			\gamma_{T}\pqty{\lambda} = \lambda\pqty{\lambda - 1}^{n-1} \\
			\gamma_{T}\pqty{-1} = \pqty{-1}^n \pqty{2}^{n-1}
		\end{gather*}
	\end{Example*}
	\begin{Example*}
		Let $K_n$ denote the \emph{complete graph} with $n$ vertices. Since every vertex is adjacent to every other vertex, the number of colours one can use for the rest of the vertices reduces by 1 after colouring each vertex. Therefore, its chromatic polynomial and number of acyclic orientations are:
		\begin{gather*}
			\gamma_{K_n}\pqty{\lambda} = \lambda\pqty{\lambda - 1}\dots\pqty{\lambda - n + 1} \\
			\gamma_{K_n}\pqty{-1} = \pqty{-1}^n n!
		\end{gather*} 
	\end{Example*}
	\begin{Theorem}[Stanley's Theorem]
		The chromatic polynomial of a graph $G$ with $n$ vertices has the following property:
		\begin{gather*}
			\gamma_{G}\pqty{-1} = \sum_{k \geq 1}^n \gamma'_G(k) \pqty{-1}^k = \pqty{-1}^n \bqty{\mathrm{\#\;of\;acyclic \;orientations\;of\;} G}
		\end{gather*}
	\end{Theorem}
\end{document}