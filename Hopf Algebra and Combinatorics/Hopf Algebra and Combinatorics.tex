% Hopf Algebras and Combinatorics

\documentclass{article}
\usepackage[paperwidth=5.9in, paperheight=9in, top = 20mm, bottom = 15mm, left=8mm, right = 8mm]{geometry}

\usepackage{natbib}

\usepackage{amsmath, amsthm, amssymb}
\setcounter{tocdepth}{3}
\usepackage{graphicx}
\usepackage{physics}

\usepackage[english]{babel} % Pour adopter les règles de typographie française
\usepackage[T1]{fontenc} % Pour que les lettres accentuées soient reconnues

\usepackage{url}

\newtheorem{Theorem}{Theorem}
\newtheorem{Lemma}{Lemma}

\theoremstyle{definition}
\newtheorem{Definition}{Definition}
\newtheorem*{Definition*}{Definition}
\newtheorem{Example}{Example}
\newtheorem*{Example*}{Example}

\theoremstyle{remark}
\newtheorem*{Remark*}{Remark}

\DeclareMathOperator{\Mat}{Mat}

\setlength{\parindent}{0pt}

\begin{document}

\title{\textbf{Hopf Algebras and Combinatorics}}

\author{\small Arjit Seth \\ \small Manipal Institute of Technology, Manipal University}
\date{}
\maketitle

\renewcommand{\abstractname}{Context}
\begin{abstract}
These are notes created based on the lectures from San Francisco State University under Prof. Federico Ardila. Hopf algebras are cool. They apparently have applications in quantum field theory, pertinent to Feynman diagrams.
\end{abstract}

\begingroup
\let\clearpage\relax
\tableofcontents
\endgroup

\medskip
\medskip

\section{Introduction}\label{sec:Intro}
A Hopf algebra is a complicated mathematical structure with a definition involving lots of properties and scary commutative diagrams. It fundamentally invokes the tensor product, and it is difficult to construct as a complete definition in one sitting, so we must do it in pieces.

\begin{Definition}[Hopf algebra]\label{def:HopfAlg}
A \emph{Hopf algebra} $H$ is a $\mathbb{K}$-vector space with five operations:
\begin{center}
\begin{tabular}{rl}
Multiplication: &   $ m\colon H \otimes H \to H$\\
Unit: & $u\colon \mathbb{K} \to H $ \\
Comultiplication: & 	$ \Delta\colon H \to H \otimes H$\\
Counit: & $\epsilon\colon H \to \mathbb{K}$ \\
Antipode: &  $ S\colon H \to H $
\end{tabular}
\end{center}
\end{Definition}

These definitions invoke lots of commutative diagrams that are difficult to \TeX, so they'll be shown later once a better understanding is developed. Now we must preliminarily define the tensor product.

\begin{Definition}[Tensor product]\label{def:TensorProd}
Let $\mathbb{K}$ be a field, and let $V$ and $W$ be vector spaces over $\mathbb{K}$. The tensor product $ V\otimes W $ is a vector space over $ \mathbb{K} $ generated by vectors $v \otimes w$, $v \in V$, $w \in W$, and satisfying the following properties:
\begin{center}
	Distributivity over addition -- \hfill $ (v + v') \otimes (w + w') = v \otimes w + v \otimes w' + v' \otimes w + v' \otimes w'$
	Scalar multiplication independent of two arguments -- \hfill $ \lambda(v \otimes w) = \lambda v \otimes w = v \otimes \lambda w$
\end{center}
\end{Definition}

The combination of these is a property called \emph{bilinearity}. Note that this vector space is much larger than the product space $V \times W$:
\begin{align*}
\dim U \times V & = \dim U + \dim V\\
\dim U \otimes V & = \dim U \dim V
\end{align*}
This is intuitively obvious because the tensor product defines an \emph{actual} product between elements of $V$ and $W$, instead of a restricted component structure induced by a Cartesian product, which constrains manipulations to $V$ and $W$ independently.

\begin{Example*}[Tensor product]
If $ \qty{v_i}_{i \in I}$ and $ \qty{w_j}_{j \in J} $ are bases of $V$ and $W$, then $ \qty{v_i \otimes w_j \mid i \in I, \; j \in J}$ is a basis for $V \otimes W$. Let $ \qty{v_i}$ and $\qty{w_j} $ be the standard bases in two and three dimensions, represented as column and row vectors respectively. One element of the basis for the tensor product is:
\begin{gather*}
	v_1 \otimes w_3 = \mqty[1 \\ 0] \otimes \mqty[0 & 0 & 1]  = \mqty[0 & 0 & 1 \\ 0 & 0 & 0]
\end{gather*}
It is clear that this construction develops the standard basis for $V_{2} \otimes W_{3}$.
\end{Example*}

\begin{Example*}[Polynomial ring and matrices] Let $ V = \mathbb{R}[x]$ and $ W = \Mat_{2\times 2}(\mathbb{R})$. A formal expression of an element in $V \otimes W $ is, for example, $(2 + 2x) \otimes \mqty[0 & 1 \\ 1 & 0]$.
\end{Example*}

\subsection{Motivating Examples for Hopf Algebras}
\begin{Example}[Groups]
Let $G$ be a group and $\mathbb{K}$ a field. To allow multiplication of scalars into the group, define the group ring:
$$\mathbb KG = \qty{\, \sum_{i = 1}^{n}\lambda_i g_i \biggm| \lambda_i \in \mathbb{K},\; g_i \in G \,} $$
Define a multiplication between elements as:
$$ m(g \otimes h) = \pqty{\sum_{i = 1}^{n}\lambda_i g_i}\pqty{\sum_{j = 1}^{n}\mu_j h_j} = \sum_{ij}  \lambda_i \mu_j(g_i h_j) $$
Define a comultiplication as: $ \Delta(g) = g \otimes g $. Extending this linearly:
$$ \Delta\pqty{\sum_{i = 1}^n \lambda_ig_i} = \sum_{i = 1}^n \lambda_i(g_i \otimes g_i)$$
These definitions do give a Hopf algebra according to the lecturer.
\end{Example}

\begin{Remark*}
The comultiplication $ \Delta(g) = 1 \otimes g + g \otimes 1 + g \otimes g$ should also be valid, as the mapping \emph{should} point to all possible information of where $g$ can come from. As we will see later, the $ g \otimes g $ is an extra term in this case.
\end{Remark*}
\end{document}