% Hopf Algebras and Combinatorics

\documentclass[svgnames]{article}
\usepackage[paperwidth=6in, paperheight=8in, top = 20mm, bottom = 18mm, left=10mm, right = 10mm]{geometry}

\usepackage{amsmath, amsfonts, amssymb, amsthm}
\setcounter{tocdepth}{3}
\usepackage[all]{xy}
\usepackage{physics}
\usepackage{mathtools}
\usepackage{xspace}
\usepackage{enumitem}

\usepackage{url}
\usepackage{hyperref}
\hypersetup{colorlinks, linkcolor = [RGB]{66, 128, 128}, urlcolor = darkgray, linktocpage = true}

\usepackage{newpxmath}
\usepackage{charter}
\usepackage[T1]{fontenc}

\usepackage{parskip}

\newtheorem{Theorem}{Theorem}
\newtheorem{Lemma}[Theorem]{Lemma}
\newtheorem{Conjecture}[Theorem]{Conjecture}
\newtheorem{Corollary}[Theorem]{Corollary}

\theoremstyle{definition}
\newtheorem{Definition}[Theorem]{Definition}
\newtheorem*{Definition*}{Definition}
\newtheorem{Example}[Theorem]{Example}
\newtheorem*{Example*}{Example}

\theoremstyle{remark}
\newtheorem{Question}{Question}
\newtheorem{Observation}{Observation}
\newtheorem*{Note*}{Note}

\DeclareMathOperator{\Aut}{Aut}
\DeclareMathOperator{\Sp}{Sp}
\renewcommand{\th}{\textsuperscript{th}\xspace}

\begin{document}
\title{\textbf{The Spread of a Finite Group}}
\author{\small Vinay Madhusudanan}
\date{\scriptsize November 5, 2020}
\maketitle

\begin{abstract}
Notes of Scott Harper's talk on the spread of a finite group: ``Many interesting and surprising results have arisen from studying generating sets for groups. For example, every finite simple group has a generating pair, and moreover Guralnick and Kantor proved that in a finite simple group every nontrivial element is contained in a generating pair. This talk will focus on recent work with Burness and Guralnick, that completely classifies the finite groups where every nontrivial element is contained in a generating pair and thus settles a 2008 conjecture of Breuer, Guralnick and Kantor. I will also give a graph theoretic interpretation of the topic, highlight how our work answers a 1975 question of Brenner and Wiegold and discuss what is known for infinite groups.''
\end{abstract}

\begingroup
\let\clearpage\relax
\tableofcontents
\endgroup

\section{Basics}
\begin{Definition}
The \emph{spread} $S(G)$ of a group $G$ is the greatest integer $k$ such that $\forall x_1, \ldots, x_k \in G$, $\exists g \in G \setminus 1$ satisfying
\begin{equation*}
	\langle x_1, g \rangle = \cdots = \langle x_k, g \rangle = G.
\end{equation*}
\end{Definition}

\begin{Lemma}
If $S(G) \ge 1$ then $G/N$ is cyclic for all $1 \ne N \unlhd G$.
\end{Lemma}
\begin{proof}
Let $1 \ne N \unlhd G$, and let $1 \ne n \in N$. Then $\exists g \in G$ such that $\langle n, g \rangle = G$.
Thus, $G/N = \langle nN, gN \rangle = \langle gN \rangle$ is cyclic.
\end{proof}

\begin{Conjecture}
Let $G$ be finite. Then $S(G) \ge 1$ iff $G/N$ is cyclic for all $1 \ne N \unlhd G$.
\end{Conjecture}

\begin{Theorem}
Let $G$ be a finite simple group. Then $S(G) \ge 2$.
\end{Theorem}

\begin{Theorem}
Let $G$ be a finite soluble group. Then $S(G) \ge 2$ iff $S(G) \ge 1$ iff $G/N$ is cyclic for all $1 \ne N \unlhd G$.
\end{Theorem}

\begin{Question}
When is $S(G) = 1$? Are there finitely many such $G$?
\end{Question}

\begin{Question}
$S(\operatorname{PSL_2(q)}) = ?$, where $q \equiv 3 \mod 4$.
\end{Question}

\subsection{Generating graphs}
The \emph{generating graph} $\Gamma(G)$ of a group $G$ has vertex set $G\setminus1$ and edges $(g, h)$ for each pair $g, h \in G$ such that $\langle g, h \rangle = G$.
\begin{Example}$\Gamma(D_8)$
\begin{equation*}
\xymatrix{
	b \ar@{-}[dd] \ar@{-}[rd] \ar@{-}[rrrd] \ar@{-}[rrrr] 
		&&&& ab \ar@{-}[llld] \ar@{-}[ld] \ar@{-}[dd]\\
	& a & \phantom{a}a^2 & a^3 & \\
	a^3b \ar@{-}[u] \ar@{-}[ru] \ar@{-}[rrru] \ar@{-}[rrrr]
		&&&& a^2b \ar@{-}[lllu] \ar@{-}[lu] \ar@{-}[u]	
}
\end{equation*}
\end{Example}

\section{Results}
\begin{Theorem}[Burness, Guralnick, Harper]
Let $G$ be finite. Then $S(G) \ge 2$ iff $S(G) \ge 1$ iff $G/N$ is cyclic for all $1 \ne N \unlhd G$.
\end{Theorem}
\begin{Corollary}
There is no finite group of spread $1$.
\end{Corollary}

\subsection{Infinite groups}
In general, $G/N$ is cyclic for all $1 \ne N \unlhd G$ does not imply that $S(G) \ge 1$.

\begin{Observation}
Let $G$ be an infinite soluble group. Then $G/N$ is cyclic for all $1 \ne N \unlhd G$ $\implies$ $G$ is cyclic $\implies$ $S(G) = \infty$.
\end{Observation}

\begin{Theorem}
$S(V) \ge 1$, where $V$ is the Thompson group (a finitely presented infinite simple group).
\end{Theorem}

\subsection{Proof outline of Theorem (BGH)}
\begin{enumerate}
\item Reduction.

Let $G$ be a finite insoluble group such that $G/N$ is cyclic for all $1 \ne N \unlhd G$.

Then $G = \langle T_1 \times \cdots \times T_k, x \rangle$, where
\begin{align*}
	T_i \cong T\, \text{a non-Abelian simple group} \\
	x = (y, 1, \ldots, 1)(1, \ldots, k) \\
	y \in \Aut T.
\end{align*}

??? $A = \langle T, y \rangle \le \Aut T$ is an almost simple group.

\begin{Theorem}[Half of BGH]
$S(A) \ge 2 \implies S(G) \ge 2$.
\end{Theorem}

\item Almost simple groups.
\begin{enumerate}[label=(\alph*)]
	\item How to show $S(G) \ge k$.
	\item How to do this for finite simple groups of Lie type.
	\item How to go from simple to almost simple.
\end{enumerate}
\begin{enumerate}[label=(\alph*)]
	\item $G = S_n$, $n = 2m + 1$.
	
	$s = (1, \ldots, m)(m + 1, \ldots, 2m + 1)$.
	
	$H = S_m \times S_{m + 1}$ is the unique maximal subgroup of $G$ containing $s$.
	
	$\langle x, s^g \rangle = G \iff x \notin H^g$ ???$\implies$??? $S(G) \ge 1$.
	
	More complicated. Probabilistic method.
	
	\item $G = \Sp_n(q)$, $q = 2^b$.
	
	$s = \begin{pmatrix}
		A &\\ & B
	\end{pmatrix}$
	where $A$ is $m \times m$ and $B$ is $(n - m) \times (n - m)$.
	
	Choose $m$ well so that the only maximal subgroups of$G$ that contains $s$ are $H = \Sp_m(a) \times \Sp_{n-m}(a)$ and $K = O_n^*(q)$.
	
	???
	
	``Shintani descent''
	
\end{enumerate}
\end{enumerate}

\end{document}