% The cycle polynomial of permutation groups

\documentclass{article}
\usepackage[paperwidth=6in, paperheight=8in, top = 20mm, bottom = 18mm, left=10mm, right = 10mm]{geometry}

\usepackage{natbib}

\usepackage{amsmath, amsthm, amssymb}
\setcounter{tocdepth}{3}
\usepackage{graphicx}
\usepackage{physics}
\usepackage{tikz}

\usepackage{charter}
\usepackage[T1]{fontenc}

\usepackage{url}

\newtheorem{Theorem}{Theorem}
\newtheorem{Lemma}{Lemma}

\theoremstyle{definition}
\newtheorem{Definition}{Definition}
\newtheorem*{Definition*}{Definition}
\newtheorem{Example}{Example}
\newtheorem*{Example*}{Example}
\newtheorem{Proposition}{Proposition}
\newtheorem*{Proposition*}{Proposition}

\theoremstyle{remark}
\newtheorem*{Remark*}{Remark}

\newtheoremstyle{underline}% name
{}        % Space above, empty = `usual value'
{}              % Space below
{}              % Body font
{}    % Indent amount (empty = no indent, \parindent = para indent)
{}              % Thm head font
{:}             % Punctuation after thm head
{1.5mm}         % Space after thm head: \newline = linebreak
{\underline{\thmname{#1}\thmnumber{ #2}\thmnote{(#3)}}}  % Thm head spec

\theoremstyle{underline}
\newtheorem*{Multiplication*}{Multiplication}

\theoremstyle{underline}
\newtheorem*{Comultiplication*}{Comultiplication}

\DeclareMathOperator{\Mat}{Mat}
\newcommand{\Mod}[1]{\ (\mathrm{mod}\ #1)}

\setlength{\parindent}{0pt}

\begin{document}

\title{\textbf{The Cycle Polynomial of Permutation Groups}}

\author{\small Arjit Seth and M. Vinay \\ \small Manipal Institute of Technology, Manipal University}
\date{}
\maketitle

\renewcommand{\abstractname}{Context}
\begin{abstract}
These are notes created to develop an understanding of the cycle polynomial, conceptualised by Peter J. Cameron and Jason Semeraro.
\end{abstract}

\begingroup
\let\clearpage\relax
\tableofcontents
\endgroup

\medskip
\medskip

\section{Introduction}\label{sec:Intro}
The cycle polynomial of a finite permutation group is an interesting construction.

\begin{Definition}[Cycle polynomial]\label{def:CycPol}
The \emph{cycle polynomial} of a permutation group $G$ acting on a set $\Omega$ of size $n$ is defined as:
\begin{gather*}
	F_{G}(x) = \sum_{g \in G} x^{c(g)}
\end{gather*}
where $c(g)$ is the number of cycles of $g$ on $\Omega$, including fixed points.
\end{Definition}

\begin{Example*}
Consider the permutation group $\qty{e, (1 \, 2), (3 \, 4), (1 \, 2)(3 \, 4)} \cong V_4$ acting on the set $\qty{1, 2, 3, 4}$. Its cycle polynomial is:
\begin{gather}
	F_{V_4}(x) = x^4 + 2x^3 + x^2 
\end{gather}
\end{Example*}

\begin{Remark*}
The cycle polynomial will never have a constant term, since the minimum number of cycles of a permutation group can have is 1. It is obviously a polynomial of $n^{th}$ order because of the identity permutation.  
\end{Remark*}

\begin{Proposition}
If $a$ is an integer, then $F_G(a)$ is a multiple of $|G|$.
\end{Proposition}

\begin{proof}

\end{proof}

\end{document}